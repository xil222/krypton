%!TEX root = <main.tex>
\documentclass[10pt, sigconf]{acmart}
\usepackage[font=small,labelfont=bf]{caption}

\usepackage{graphicx,xspace,verbatim,comment}
\usepackage{hyperref,array,color,balance,multirow}
\usepackage{balance,float,url,amsfonts,alltt}
\usepackage{mathtools,rotating,amsmath,amssymb}
\usepackage{color,ifpdf,fancyvrb}
\usepackage{etoolbox,listings,subcaption}
\usepackage{bigstrut,morefloats,pbox}
\usepackage{amsmath}
\usepackage{algorithm}
\usepackage[noend]{algpseudocode}
\usepackage{booktabs}
\usepackage{bm}

\newtheorem{theorem}{Theorem}[section]
\newtheorem{proposition}{Proposition}[section]
\newtheorem{corollary}[theorem]{Corollary}
\newtheorem{lemma}[theorem]{Lemma}
\newtheorem{definition}{Definition}[section]

\DeclarePairedDelimiter\ceil{\lceil}{\rceil}
\DeclarePairedDelimiter\floor{\lfloor}{\rfloor}

\newcommand{\eat}[1]{}
\newcommand{\red}{\textcolor{red}}
\newcommand{\system}{\textsc{Krypton}}

\makeatletter
\def\BState{\State\hskip-\ALG@thistlm}
\makeatother 

\newenvironment{packeditems}{
\begin{itemize}
  \setlength{\itemsep}{1pt}
  \setlength{\parskip}{0pt}
  \setlength{\parsep}{0pt}
}{\end{itemize}}

\newenvironment{packedenums}{
\begin{enumerate}
  \setlength{\itemsep}{1pt}
  \setlength{\parskip}{0pt}
  \setlength{\parsep}{0pt}
}{\end{enumerate}}


\newcolumntype{P}[1]{>{\centering\arraybackslash}p{#1}}

\DeclareMathOperator*{\argmin}{arg\,min}

% \setcopyright{none}
% \settopmatter{printacmref=false}
% \renewcommand\footnotetextcopyrightpermission[1]{} % removes footnote with 
\pagestyle{plain} % removes running headers

\begin{document}
\emergencystretch 1em

%\title{\textsc{Krypton}: Accelerating Occlusion based Deep CNN\\ Explainability Workloads}
\title{Incremental and Approximate Inference for\\Faster Occlusion-based Deep CNN Explanations}

%\author{Anonymous Author(s)}
% \affiliation{
  % \institution{University of California, San Diego}
% }
% \email{@eng.ucsd.edu}


\begin{abstract}
Deep Convolutional Neural Networks (CNNs) now match human accuracy in many image recognition tasks. This has lead to increasing adoption of deep CNNs in e-commerce, radiology, and other domains. Naturally, ``explaining'' CNN predictions is a key concern for many users. Since the internal working of CNNs are unintuitive for non-technical users, occlusion-based explanations are popular for determining which parts of an input image contribute the most to a given prediction. One occludes a region of the image using a patch and moves this patch across the image to yield a heatmap of changes to the prediction probability. Alas, this approach is computationally expensive due to the large number of re-inference requests produced, which could waste human time and raise resource costs. In this paper, we resolve this issue by casting occlusion-based CNN explanations as a new instance of the incremental view maintenance problem. We create a novel and comprehensive algebraic framework for incremental CNN inference that combines materialized views with multi-query optimization to avoid computational redundancy across re-inference requests. We then introduce two approximate inference optimizations that exploit the semantics of CNNs and the occlusion task to further reduce runtimes. We prototype our ideas in Python to create a tool we call \textsc{Krypton} that can support both CPUs and GPUs. Experiments with real data and CNNs show that \textsc{Krypton} reduces runtimes by up to 5x (14x) to produce exact (approximate) heatmaps without raising resource requirements.
\end{abstract}

\maketitle

%!TEX root = <main.tex>
\section{Introduction}
Deep Convolution Neural Networks (CNNs) \cite{alexnet, vggnet, resnet, inception} have revolutionized the computer vision field with even surpassing human level accuracy in some of the image recognition challenges such as ImageNet~\cite{imagenet}.
% Many of these successful pre-trained CNNs from computer vision challenges have been successfully re-purposed to be used in other real-world image recognition tasks using a paradigm called \textit{transfer learning} \cite{transfer-learning-factors}.
% In transfer learning, instead of training a CNN from scratch, one uses a pre-trained Deep CNN, e.g., ImageNet trained VGG, and fine tune it for the target problem using the target training dataset.
% This approach avoids the need for a large training datasets, computational power, and time which is otherwise a bottleneck for training a CNN from scratch.
As a result, there is wide adoption of Deep CNN technology in a variety of real-world image recognition tasks in several domains including healthcare \cite{kermany2018identifying, islam2017abnormality}, agriculture \cite{mohanty2016using}, security \cite{arbabzadah2016identifying}, and sociology \cite{wang2017deep}.
Remarkably, United States Food and Drug Administration Agency (US FDA) has already approved the use of Deep CNN based technologies for identifying diabetic retinopathy, an eye disease found in adults with diabetes \cite{fdaretinopathy}.
It is expected that this kind of decision support systems will help the human radiologists in fulfilling their workloads efficiently, such as functioning as a cross-checker for the manual decisions and also to prioritize potential sever cases for manual inspection, and provide a remedy to the shortage of qualified radiologists globally \cite{radiologistshortage}.

\begin{figure}[t]
  \includegraphics[width=\columnwidth]{./images/krypton_overview}
  \caption{(a) Using CNNs for predicting Diabetic Retinopathy from OCT images. (b) Occluding parts of the OCT image changes the predicted probability for the disease. (c) By changing the position of the occlusion patch a sensitivity heat map is produced.}
  \label{fig:krypton_overview}
\end{figure}

Despite their many success stories, one of the major criticisms for deep CNNs and deep neural networks, in general, is the black-box nature of how they make predictions.
In order to apply deep CNN based techniques in critical applications such as healthcare, the decisions should be explainable so that the practitioners can use their human judgment to decide whether to rely on those predictions or not \cite{jung2017deep}.
One of the widely used approach for explaining CNN predictions is the \textit{occlusion based explanations} (OBE) approach \cite{zeiler2014visualizing}.
In OBE experiments a square patch, usually of black or gray color, is used to occlude parts of the image and record the change in the predicted label probability as shown in Figure~\ref{fig:krypton_overview} (b).
By changing the position of the occlusion patch a sensitivity heat map for the predicted label, similar to the one shown in Figure~\ref{fig:krypton_overview} (c), can be generated.
Using this heat map, the regions in the image which are highly sensitive (or highly contributing) to the predicted class can be identified (corresponds to red color regions in the sensitivity heat map shown in Figure~\ref{fig:krypton_overview} (c)).
This localization of highly sensitive regions then enables the practitioners to get an intuition of the the prediction process of the deep CNN.

% If the occlusion experiment is performed in interactive mode, the human operator has the option of picking the occlusion patch positions by marking a region on a visual interface.
% For example, if the scenario shown in Figure \ref{fig:krypton_overview} is performed in interactive mode, the human operator who understands OCT images will start evaluating the image from the central region where she expects the pathological region to most likely to be.
% In the non-interactive mode, which is also the most common mode of performing occlusion experiments due to the high runtimes which are not amenable for interactive performance, the heat map values are evaluated for all possible occlusion patch positions.


% \textbf{\textit{Example:}} Consider a radiologist who is examining Optical Coherence Tomography (OCT) images of the retina to identify potential diabetic retinopathy patients.
% The radiologist is recently given access to a Deep CNN based clinical decision support system (CDSS) to identify potential images with diabetic retinopathy. 
% It predicts the probability whether a retina image depicts a diabetic retinopathy case.
% She uses the CDSS for two main purposes: 1) as a cross checker while manually inspecting the retinal images, and 2) to prioritize potentially sever cases from a backlog of retina images.
% In both situations in addition to predicting the existence of the disease, she would like to have an explanation for the basis on which the CDSS makes the prediction, using the occlusion based explainability approach, to decide whether the pathological regions identified by the CDSS are correct and to ultimately whether to rely on the CDSS decision. Similar examples arise in number of other heal care applications such as chest X-ray examination for identifying pneumonia cases and X-ray based child bone age assessment.

Alas, this approach creates a new bottleneck: OBE experiments are highly compute intensive and time consuming due to the large number of re-inference requests that needs to be performed.
The current approach to perform OBE experiments is to generate a large number of modified versions of the original image with each image corresponding to a specific occlusion patch position and perform CNN inference for those images using a tool like PyTorch \cite{ketkar2017introduction}.
In extreme cases this approach can generate more than 500,000 different images and can take up to more than one hour to complete, even by using a GPU~\cite{zintgraf2017visualizing}.
Such long execution times hinders the data scientist's ability to analyze CNN predictions and also reduces their productivity.

\textit{In this work we apply database inspired optimizations to the OBE workload to reduce both the computational cost and the runtime}.
Our work is motivated by the simple yet crucial observation: different occluded instances of the original image are \textit{not} independent -- when performing CNN inference corresponding to each individual occluded image significant portion of redundant computation can be avoided.
This observation leads us to a classical database systems-style concern: \textit{view maintenance}.
In database parlance, the current approach of performing CNN inference for each occluded instance of the image independently can be considered as a form of \textit{complete view maintenance}, where a ``view'' is a layer of CNN features.
This approach wastes runtimes due redundancy in CNN inference computations across occluded instances of the original image. 

We formalize the dataflow of the layers of a CNN and create a novel and comprehensive algebraic framework for \textit{incremental inference} of CNN which combines \textit{incremental view maintenance} (IVM) \cite{chirkova2012materialized,gupta1995maintenance,levy1995answering} with \textit{multi-query optimization} (MQO) to avoid computational redundancy across re-inference requests.
Using our algebraic framework we theoretically characterize how much speedups one can expect from incremental inference for different CNN models.
To the best of our knowledge, this is the  first known instance of fusing an IVM-style techniques with an MQO-style technique for optimizing CNN inference.

We then introduce \textit{approximate inference} optimizations that exploit a capability of human perception: tolerance of some degradation in visual quality of the heat maps.
We build upon our incremental inference optimizations to create two novel approximate inference optimizations that trade off the quality of the generated heat map in a user tunable manner to accelerate OBE.
Our approximate inference optimizations are inspired by the \textit{approximate query processing} (AQP) techniques used for answering statistical analytical queries in RDBMs.
However, our focus here is on minimizing the perceivable differences of the generated heat maps where as in traditional AQP the focus is on minimizing the statistical error.
We also combine AQP and MQO style techniques to reduce OBE runtime.


% This will also make occlusion experiments more amenable for interactive diagnosis of CNN predictions.

% Due to the overlapping nature of how the Convolution kernel operates (details to follow in Section \ref{sec:preliminaries}), the size of the modified patch will start growing as it progresses through more layers in a CNN and the amount of redundant computations will reduce.
% However, at deeper layers, the effect over the patch coordinates which are radially further away from the center of the occlusion patch position will be diminishing.
% Our second optimization is based on this observation where we apply a form of \textit{approximate inference} which applies a threshold to limit the growth of the updating patch. 
% By applying propagation thresholds, a significant amount of computation redundancy can be retained.
% We refer to this optimization as \textit{projective field thresholding}.

% The third optimization is also a form of \textit{approximate inference} which is applicable only in the context of non-interactive mode.
% In most occlusion experiment use cases, such as in medical imaging, the object or pathological region of interest is contained in a relatively small region of the image.
% In such situations, it is unnecessary to inspect the original image at the same high resolution of striding the occluding patch few pixels at a time, at all possible occlusion patch positions.
% In this approach first, a low-resolution heat map is generated using a larger stride value with a relatively low computational cost.
% Only the interesting regions will be then inspected further with a smaller stride to produce a higher resolution output.
% In the interactive mode, as the human operator will be actively picking a set of occlusion patch positions for the system to evaluate this optimization will not be applicable.
% We refer to this optimization as \textit{adaptive drill-down}.

% Unlike the \textit{incremental inference} approach which is exact, \textit{projective field thresholding} and \textit{adaptive drill-down} are approximate approaches. They essentially trade-off accuracy of the generated sensitivity heat map compared to the original, in favor of faster execution.
% These changes in accuracy in the generated heat map can be visible all the way from quality differences which are almost indistinguishable to the human eye to drastic structural differences, depending on the level of approximation.
% This opens up an interesting trade-off space of quality/accuracy versus runtime. \system~ provides user configurable tuning parameters for easily picking an operational point on this quality-runtime trade-off space.

We prototype our ideas as a system we call \system~ on top of PyTorch deep learning toolkit by adding custom implementations for incremental and approximate inference operations.
It currently supports VGG16, ResNet18, and InceptionV3 both on CPU and GPU environments, which are three widely used Deep CNN architectures.
We evaluate our system on three real-world datasets, 1) retinal optical coherence tomography dataset (OCT), 2) chest X-Ray, and 3) more generic ImageNet dataset.
While we have implemented \system~ on top of PyTorch toolkit, our work is largely orthogonal to the choice of the deep learning toolkit; one could replace PyTorch with TensorFlow, Caffe2, CNTK, MXNet, or implement from scratch using C/CUDA and still benefit from our optimizations.
Overall, this paper makes the following contributions:

\begin{itemize}
	\item To the best of our knowledge, this is the first paper to study the incremental inference and approximate inference optimizations for the OBE workload from a systems standpoint.

	\item We develop a comprehensive algebraic framework for incremental inference of CNN which combines IVM with MQO to avoid computational redundancy across re-inference requests.

	\item We also introduce two approximate inference optimizations that exploit the characteristics of human perception to further reduce runtimes of the OBE workload.

	\item Using a prototype system call \system, we present an extensive empirical evaluation of the benefits of our optimizations. Overall, \system~  can result in speedups up to 5x (14x) to produce exact (approximate) heat maps.

\end{itemize}

\vspace{2mm}
\noindent \textbf{Outline.} The rest of this paper is organized as follows.
Section 2 formally define the problem, explains our assumptions, and formalizes the dataflow of layers of a CNN.
Section 3 provides a theoretical characterization of potential speedups for different CNN models and presents our novel algebraic framework for performing incremental inference for OBE.
Section 4 presents our approximate inference optimizations.
Section 5 presents the experimental evaluation.
We discuss other related work in Section 6 and conclude in Section 7.


\section{Background}

\vspace{2mm}
\noindent \textbf{Deep CNNs.} CNNs are a type of neural networks specialized for image data.
They exploit spatial locality of information in image pixels to construct a hierarchy of parametric feature extractors and transformers organized as layers of various types: \textit{convolutions}, which use image
filters from graphics, except with variable filter weights, to extract features; \textit{pooling}, which subsamples features in a spatial
locality-aware way; \textit{non-linearity} to apply a non-linear
function (e.g., ReLU) to all features; and \textit{fully connected},
which is a multi-layer perceptron.
A ``deep'' CNN just stacks such layers many times over.
Popular deep CNN model architectures include AlexNet \cite{alexnet}, VGG \cite{vggnet}, Inception, ResNet, SqueezeNet, and MobileNet.
In this work, the discussion and evaluation is focused on VGG, ResNet and SqueezeNet which are three widely used CNN models in real world use cases.
Nevertheless, our work is orthogonal to how CNNs are designed and the proposed approaches can be easily extended to any architecture.

\vspace{2mm}
\noindent \textbf{Deep CNN Explainability} With image classification models, natural question is if the model is truly identifying objects in the image or just using surrounding or other objects for making false predictions.
The various approaches used to explain CNN predictions can be broadly divided into two categories, namely gradient based and perturbation based approaches. Gradient based approaches generate a sensitivity map by computing the partial derivatives of model output with respect to every input pixel via back propagation.
In perturbation based approaches the output of the model is observed by masking out regions in the input image and there by identify the sensitive regions. The most popular perturbation based approach is occlusion experiments which was first introduced by Zeiler et. al. \cite{zeiler2014visualizing}.
Even though gradient approaches require only a single forward inference and a single backpropagation to generate the sensitivity map, the output may not be very intuitive and hard to understand because the salient pixels tend to spread over a very large are of the input image.
As a result in most real world use cases such as in medical imaging, practitioners tend to use occlusion experiments as the preferred approach for explanations as they produce high quality fine grained sensitivity maps despite being time consuming.

Over the years there has been several modifications proposed to the original occlusion experiment approach. More recently Zintgraf. et. al. \cite{zintgraf2017visualizing} proposed a variation to the original occlusion experiment approach named \textit{Prediction Difference Analysis}. In this method instead of masking with a grey or black patch, samples from surrounding regions in the image are chosen as occlusion patches.
In our work we mainly focus on the original occlusion experiment method. But, the methods and optimizations proposed in our work are readily applicable to more advanced occlusion based explainability approaches.

%!TEX root = <main.tex>
\section{Preliminaries and Overview}\label{sec:preliminaries}
In this section first we formally state the problem and explain our assumptions.
Then we formalize the internals of some of the layers in a Deep CNN for the purpose of proposing our incremental inference approach in Section 4.
Finally we briefly explain the Structural Similarity Index (SSIM) which is used to quantify the quality of the generated sensitivity heatmaps.

\subsection{Problem Statement and Assumptions}

\begin{table}[t]
  \centering
  \caption{Symbols used in the Preliminaries Section}
  \scalebox{0.8}{\begin{tabular}{p{2cm}p{7.5cm}}
    \toprule
    \textbf{Symbol} & \textbf{Meaning}\\
    \midrule \midrule
    $\mathcal{I} (\mathcal{I}_{img})$ & Input activation volume (Input Image)\\
    \midrule
    $\mathcal{O}$ & Output activation volume\\
    \midrule
    $C_{\mathcal{I}},H_{\mathcal{I}},W_{\mathcal{I}}$ & Depth, height, and width of Input\\
    \midrule
    $C_{\mathcal{O}},H_{\mathcal{O}},W_{\mathcal{O}}$ & Depth, height, and width of Output\\
    \midrule
    $\mathcal{K}_{conv}$ & Convolution filter kernels\\
    \midrule
    $\mathcal{B}_{conv}$ & Convolution bias value vector\\
    \midrule
    $\mathcal{K}_{pool}$ & Pooling filter kernel\\
    \midrule
    $H_{\mathcal{K}},W_{\mathcal{K}}$ & Height and width of filter kernel\\
    \midrule
    $S (S_x,S_y)$ & Filter kernel or occlusion patch striding amount ($S_x$ and $S_y$ corresponds to width and height dimensions)\\
    \midrule
    $P (P_x,P_y)$ & Padding amount ($P_x$ and $P_y$ corresponds to padding along width and height dimensions)\\
    \midrule
    % $Q (Q_{inc})$ & Total FLOPS count with full (incremental) inference\\
    % \midrule
    % $L$ & Set of convolution layers in a CNN\\
    % \midrule
    $\mathcal{P}$ & Occluding patch\\
    \midrule
    $\mathcal{I}^{x,y}_{img}$ & Modified image by superimposing $\mathcal{P}$ on top of $\mathcal{I}_{img}$ such that the top left corner of $\mathcal{P}$ is positioned at $x,y$ location of $\mathcal{I}_{img}$\\
    \midrule
    $M$ & Heat map produced by the occlusion experiment\\
    \midrule
    $H_M,W_M$ & Height and width of $M$\\
    \midrule
    $f$ & Fine-tuned CNN\\
    \midrule
    $C$ & Class label predicted by $f$ for the original image $\mathcal{I}_{img}$\\
    \midrule
    $\bm\circ_{x,y}$ & Superimposition operator. $A~\circ_{x,y}~B$, superimposes $B$ on top of $A$ starting off at $(x,y)$ position\\
    \bottomrule
  \end{tabular}}
\label{table:preliminaries_symbols}
\end{table}


We are given a fine-tuned CNN $f$, an image $\mathcal{I}_{img}$ on which the occlusion experiment needs to be run, the predicted class label $C$ for the image $\mathcal{I}_{img}$, an occluding patch $\mathcal{P}$ in RGB format, and patch striding amount $S$ which will be used to move the occlusion patch on the image at a time in horizontal and vertical directions. The occlusion experiment workload is to generate a 2D heat map $M$ where each individual value in $M$ corresponds to the predicted probability for $C$ by $f$ when $\mathcal{P}$ is superimposed on $\mathcal{I}_{img}$ corresponding to the position of that particular value. More precisely, we can state the workload using the following set of logical statements:


\begin{align}
\label{eqn:mheight}
W_M =&~ \lfloor(\texttt{width}(\mathcal{I}_{img}) - \texttt{width}(\mathcal{P}) + 1)/S\rfloor\\
\label{eqn:mwidth}
H_M =&~ \lfloor(\texttt{height}(\mathcal{I}_{img}) - \texttt{height}(\mathcal{P}) + 1)/S\rfloor\\
M \in&~ \mathcal{\rm I\!R}^{H_M \times W_M}\\
\forall~ x,y \in&~ [0, H_M) \times [0, W_M):\\
\label{eqn:patchimpose}
& \mathcal{I}_{img}^{x,y} \leftarrow ~\mathcal{I}_{img} ~ \bm\circ_{x,y} ~ \mathcal{P} \\
\label{eqn:outputval}
& M[x,y] \leftarrow f(\mathcal{I}_{img}^{x,y},C)
\end{align}

Step (\ref{eqn:mheight}), and (\ref{eqn:mwidth}) calculates the dimensions of the generated heat map $M$ which is dependent on the dimensions of $\mathcal{I}_{img}$, $\mathcal{P}$, and $S$.
Step (\ref{eqn:patchimpose}) superimposes $\mathcal{P}$ on $\mathcal{I}_{img}$ with its top left corner placed on the (x,y) location of $\mathcal{I}_{img}$.
Step (\ref{eqn:outputval}) calculates the output value at the (x,y) location by performing CNN inference for $\mathcal{I}_{img}^{x,y}$ using $f$ and taking the predicted probability for the label $C$.

We assume that $f$ is a CNN from a roaster of well-known CNNs (currently, VGG 16 layer version, ResNet 18 layer version, and Inception version 3).
This is a reasonable start, since most recent CNN based image recognition applications use only such well-known CNNs from model zoos \cite{caffemodelzoo, tfmodelzoo}.
Nevertheless, our work is orthogonal to the specifics of a particular architecture and the proposed approaches can be easily extended to any architecture.
We leave support for arbitrary CNNs to future work.

\subsection{Deep CNN Internals}
Input and output of individual layers in a Deep CNN except for fully-connected ones are arranged into three dimensional volumes which has a width, height, and depth.
For example an RGB input image of 224$\times$224 spatial size can be considered as an input volume having a width and height of 224 and a depth of 3 (corresponding to 3 color channels). Every non fully-connected layer will take in an input volume and transform it into another volume.
A fully-connected layer takes in a vector or flattened activation volume as input and transforms it into another vector. For our purpose these transformations can be broadly divided into three categories based on how they operate spatially:

\begin{figure*}[t]
\includegraphics[width=\textwidth]{images/cnn_simplified}
\caption{Simplified representation of selected layers of a Deep CNN. For simplicity sake addition of bias is not shown in the Conv. transformation. Notation used is explained in Table ~\ref{table:preliminaries_symbols}.}
\label{fig:cnn_simplified}
\end{figure*}

\begin{itemize}
	\item Transformations that operate at the granularity of individual  spatial locations (see Figure \ref{fig:cnn_simplified} (b)).
	\begin{itemize}
	 \item E.g. ReLU, Batch Normalization
	\end{itemize}
	\item Transformations that operate at the granularity of a local spatial context (see Figure \ref{fig:cnn_simplified} (a) and (c)).
	\begin{itemize}
	 \item E.g. Convolution, Pooling
	\end{itemize}
	\item Transformations that operate at the granularity of a global context.
	\begin{itemize}
	 \item E.g. Fully-Connected
	\end{itemize}
\end{itemize}

With incremental spatially localized updates in the input, such as placing an occlusion patch, both types of transformations that operate at the granularity of individual spatial locations and local spatial contexts provide opportunities for exploiting redundancy.
Extending point transformations to become redundancy aware is straightforward. However, with local context transformations, such as Convolution and Pooling, this extension is non-trivial due to the overlapping nature of the spatial contexts. A simplified representation of point transformations and local context transformations is shown in Figure \ref{fig:cnn_simplified}.
It also shows how different types of transformations would propagate a spatially localized change (marked in red) in the input.
We next formally define the transformations of Convolution and Pooling layers and also the relationship between input and output dimensions.

\vspace{2mm}
\noindent \textbf{Transformations that operate at the granularity of a local spatial context.} The two types of transformations that operate at the granularity of a local spatial context in a deep CNN are Convolution and Pooling.
% Convolution layers are the most important type of layer in the CNN architecture which also contributes to most of the computational cost.
Each Convolutional layer can have several (say $C_{\mathcal{O}}$) three dimensional filter kernels organized into a four dimensional array $\mathcal{K}_{conv}$ with each having a smaller spatial width $W_\mathcal{K}$ and height $H_\mathcal{K}$ compared to the width $W_{\mathcal{I}}$ and height $H_{\mathcal{I}}$ of the input volume $\mathcal{I}$, but has the same depth $C_{\mathcal{I}}$.
During inference, each filter kernel is slided along the width and height dimensions of the input and a two dimensional activation map is produced by taking element-wise product between the kernel and the input and adding a bias value $\mathcal{B}[c]$ for some c $\in$ $[0, C_{\mathcal{O}}-1]$.
These two dimensional activation maps are then stacked together along the depth dimension to produce an output volume $\mathcal{O}$ having the dimensions of ($C_{\mathcal{O}}$,$H_{\mathcal{O}}$,$W_{\mathcal{O}}$).
A simplified representation of Convolution transformation is shown in Figure \ref{fig:cnn_simplified} (a).
It can also be formally defined as follows:

\begin{align}
\begin{split}
\text{Input Volume}:&~ \mathcal{I} \in \mathcal{\rm I\!R}^{C_{\mathcal{I}} \times H_{\mathcal{I}} \times W_{\mathcal{I}}}\\
\text{Conv. Filters}:&~ \mathcal{K}_{conv} \in \mathcal{\rm I\!R}^{C_{\mathcal{O}} \times C_{\mathcal{I}} \times H_{\mathcal{K}} \times W_{\mathcal{K}}}\\
\text{Conv. Bias Vector}:&~ \mathcal{B}_{conv} \in \mathcal{\rm I\!R}^{C_{\mathcal{O}}}\\
\text{Output Volume}:&~ \mathcal{O} \in \mathcal{\rm I\!R}^{C_{\mathcal{O}} \times H_{\mathcal{O}} \times W_{\mathcal{O}}}
\end{split}
\end{align}

\begin{equation}
\label{eqn:conv_operator}
\begin{split}
\mathcal{O}[c,y,x] &= \sum_{k=0}^{C_{\mathcal{I}}} \sum_{j=0}^{H_\mathcal{K}-1} \sum_{i=0}^{W_\mathcal{K}-1} \mathcal{K}_{conv}[c, k, j, i] \\ & \quad \quad \quad \times \mathcal{I}[k,y-\floor{\frac{H_\mathcal{K}}{2}}+j,x-\floor{\frac{W_\mathcal{K}}{2}}+i] + \mathcal{B}[c]
\end{split}
\end{equation}

Pooling can also be thought as a Convolution operation with a fixed (i.e. not learned) two dimensional filter kernel $\mathcal{K}_{pool}$.
But unlike Convolution, Pooling operates independently on each depth slice of the input volume.
% The two main variations of pooling layers are max pooling (takes the maximum value from the local spatial context) and average (takes the average value from the local spatial context) pooling.
A Pooling layer takes a three dimensional activation volume $\mathcal{O}$ having a depth of $C$, width of $W_{\mathcal{I}}$, and height of $H_{\mathcal{I}}$ as input and produces another three dimensional activation volume $\mathcal{O}$ which has the same depth of $C$, width of $W_{\mathcal{O}}$, and height of $H_{\mathcal{O}}$ as the output.
Pooling kernel is generally slided with more than one pixel at a time and hence $W_{\mathcal{O}}$ and $H_{\mathcal{O}}$ are generally smaller than $W_{\mathcal{I}}$ and $H_{\mathcal{I}}$.
A simplified representation of Pooling transformation is shown in Figure \ref{fig:cnn_simplified} (c).
% Similar to Convolution, Pooling operation can be formally defined as follows:

% \begin{align}
% \text{Pool Filters}:&~ \mathcal{K}_{pool} \in \mathcal{\rm I\!R}^{H_{\mathcal{K}} \times W_{\mathcal{K}}}
% \end{align}

% \begin{equation}
% \label{eqn:pool_operator}
% \begin{split}
% \mathcal{O}[c,y,x] &= \sum_{j=0}^{H_\mathcal{K}-1} \sum_{i=0}^{W_\mathcal{K}-1} \mathcal{K}_{pool}[j, i] \\ & \quad \quad \quad \times \mathcal{I}[c,y-\floor{\frac{H_\mathcal{K}}{2}}+j,x-\floor{\frac{W_\mathcal{K}}{2}}+i]
% \end{split}
% \end{equation}


\vspace{2mm}
\noindent \textbf{Relationship between Input and Output Spatial Sizes.}
The output volume's spatial sizes $W_{\mathcal{O}}$ and $H_{\mathcal{O}}$ are determined by the spatial sizes of the input volume $W_{\mathcal{I}}$ and $H_{\mathcal{I}}$, spatial sizes of the filter kernel $W_\mathcal{K}$ and $H_\mathcal{K}$ and two other parameters: \textbf{stride} $S$ and \textbf{padding} $P$.
Stride is the amount of pixel values used to slide the filter kernel at a time when producing a two dimensional activation map.
It is possible to have two different values with one for the width dimension $S_x$ and one for the height dimension $S_y$.
Generally $S_x \leq W_\mathcal{K}$ and $S_y \leq H_\mathcal{K}$.
In Figure \ref{fig:cnn_simplified} Conv. transformation uses a stride value of 1 and Pool transformation uses a stride value of 2 for both dimensions.
Sometimes in order to control the spatial size of the output activation map to be same as the input activation map, one needs to pad the input feature map with zeros around the spatial border.
Padding $P$ captures the amount of zeros that needs to be added.
Similar to the stride $S$, it is possible to have two separate values for padding with one for the width dimension $P_x$ and one for the height dimension $P_y$.
In Figure \ref{fig:cnn_simplified} Conv. transformation pads the input with one line of zeros from both dimensions.
With these parameters defined the width (similarly height) of the output activation volume can be defined as follows:

\begin{align}
\begin{split}
W_{\mathcal{O}} = (W_{\mathcal{I}} - W_\mathcal{K} + 1 + 2\times P_x)/S_x \\
% H_{\mathcal{O}} = (H_{\mathcal{I}} - H_\mathcal{K} + 1 + 2\times P_y)/S_y
\end{split}
\end{align}

\vspace{2mm}
\noindent \textbf{Estimating the Computational Cost of Deep CNNs.}
Deep CNNs are highly compute intensive and out of the different types of layers Conv. layers contributes to $90\%$ (or more) of the computations. One of the widely used ways to estimate the computational cost of a Deep CNN is to estimate the number of fused multiply add (FMA) floating point operations (FLOPs) required by convolution layers for a single forward inference.
% and ignore the computational cost of other layers (e.g. Pooling, Fully-Connected).

For example, applying a convolution filter having the dimensions of ($C^l_{in}$, $H^l_{\mathcal{K}}$, $W^l_{\mathcal{K}}$) to a single spatial context will require $C^l_{\mathcal{I}} \times H^l_{\mathcal{K}} \times W^l_{\mathcal{K}}$ many FLOPs, each corresponding to a single element-wise multiplication.
Thus, the total amount of computations $Q_l$ required by that layer in order to produce an output $\mathcal{O}$ having dimensions $C^l_{\mathcal{O}} \times H^l_{\mathcal{O}} \times W^l_{\mathcal{O}}$, and the total amount of computations $Q$ required to process the entire set of convolution layers $L$ in the CNN can be calculated as per Equation \ref{eqn:full_local} and ~\ref{eqn:full_all}.
However, in the case of incremental updates only a smaller spatial patch having a width $W^l_\mathcal{P}$ ($W^l_\mathcal{P}<=W^l_{\mathcal{O}}$) and height $H^l_\mathcal{P}$ ($H^l_\mathcal{P}<=H^l_{\mathcal{O}}$) is needed to be recomputed.
The amount of computations required for the incremental computation $Q^l_{inc}$ and total amount of incremental computations $Q_{inc}$ required for the entire set of convolution layers $L$ will be smaller than the above full computation values and can be calculated as per Equation ~\ref{eqn:inc_local} and ~\ref{eqn:inc_all}.

Based on the above quantities we define a new metric named \textit{theoretical speedup} R, which is the ratio between total full computational cost $Q$ and total incremental computation cost $Q_{inc}$ (see Equation ~\ref{eqn:redundancy_ratio}).
This ratio essentially acts as a surrogate for the theoretical upper-bound for computational and runtime savings that can be achieved by applying incremental computations to deep CNNs.

\begin{align}
\label{eqn:full_local}
Q^l =&~ (C^l_{\mathcal{I}} \times H^l_{\mathcal{K}} \times W^l_{\mathcal{K}}) \times (C^l_{\mathcal{O}} \times H^l_{\mathcal{O}} \times W^l_{\mathcal{O}})\\
\label{eqn:full_all}
Q =&~ \sum_{l \in L} Q^l\\
\label{eqn:inc_local}
Q_{inc}^l =&~ (C^l_{\mathcal{I}} \times H^l_{\mathcal{K}} \times W^l_{\mathcal{K}}) \times (C^l_{\mathcal{O}} \times H^l_{\mathcal{P}} \times W^l_{\mathcal{P}})\\
\label{eqn:inc_all}
Q_{inc} =&~ \sum_{l \in L} Q^l_{inc}\\
\label{eqn:redundancy_ratio}
R =&~ \frac{Q}{Q_{inc}}
\end{align}


\subsection{Estimating the Quality of Generated Approximate Heat Maps}

When applying approximate inference optimizations \system~ sacrifices the the accuracy/quality of the generated heat map in favor of faster execution.
To measure this drop of accuracy we use Structural Similarity (SSIM) Index~\cite{wang2004image} which is one of the widely used approaches to measure the \textit{human perceived difference} between two similar images.
When applying SSIM index we treat the original heat map as the reference image with no distortions and the perceived image similarity of the \system~generated heat map is calculated with reference to it.
The generated SSIM index is a value between $-1$ and $1$, where $1$ corresponds to perfect similarity.
% It is important to note that, even though SSIM index value of 1 corresponds to perfect similarity, other values doesn't necessarily imply same level of perceived quality across different image pairs.
% However, if the original images are closely similar, such as in chest X-ray images, it can be assumed that this condition will hold.
Typically SSIM index values in the range of $0.90-0.95$ are used in practical applications such as image compression and video encoding as at the human perception level they produce indistinguishable distortions.
For more details on SSIM Index method, we refer the reader to the original SSIM Index paper~\cite{wang2004image}.



\section{Optimizations}

\subsection{Incremental Inference of Convolution and Pooling Layers}\label{sec:ivm}

As explained earlier, occlusion experiments are performed by performing CNN inference on large amounts of images which are generated by masking a small square region of the input image.
When we compare the original image and a single masked image most of the pixel values are not changed. For example for an input image of size $224\times224$ pixels when using a occlusion patch of size $16\times16$ only $0.5\%$ of the pixels are different between the two images.
As a result of this performing a full convolution inference on the masked image introduces lot of redundant computations which could potentially be saved.
For example consider the simple 2D convolution example shown in Figure. \ref{fig:perturbation}.
The input feature map is square map with size $W_1=4$ and is convolved by a 2D square filter kernel of size $F=3$ with a stride of $S=1$ to produce an output feature map of size $W_2=4$.
The input feature map is also padded with zeros with a pad width of $P=1$ to ensure that both the input feature map size and the output feature map size is the same (this step is optional).
Now if we update the top left corner value in the input feature map (marked in red), it will only update 4 output values at the top left corner of the output feature map which corresponds to filter map positions on the input feature map with some overlapping with the updated input value.
Even though in this example the amount of updated values in the output feature maps is $25\%$ of the total output values, in general with larger feature map sizes the portion of updated values will be much smaller.
This analysis similarly applies to pooling layers.

\begin{figure}
  \includegraphics[width=\columnwidth]{./images/small_perturbations}
  \caption{Small perturbations in the input feature map only affect small regions in the output feature map after convolution}
  \label{fig:perturbation}
\end{figure}

More generally the propagation changes in the output feature map of a convolution or pooling layer caused by updating a patch in the input feature map is determined by the filter size $F$ and the stride $S$ of the Conv filter.
For example consider the situation shown in Figure. \ref{fig:patch_propagation}.
Assume a modified patch is placed on the input feature map which spans across $x_{in\_0}\rightarrow x_{in\_1}$ in $x$ dimension and $y_{in\_0}\rightarrow y_{in\_1}$ in y dimension.
Then the coordinates of the modified patch in the output feature map,  $x_{out\_0}\rightarrow x_{out\_1}$ in $x$ dimension and $y_{out\_0}\rightarrow y_{out\_1}$ in y dimension, can be expressed as a function of filter size $F$ and stride $S$ of the Conv/Pool filter as follows:

\begin{figure}
  \includegraphics[width=\columnwidth]{./images/patch_propagation}
  \caption{Occlusion patch propagation in Conv layers}
  \label{fig:patch_propagation}
\end{figure}

\begin{equation}
x_{out\_0} = \texttt{max}(\texttt{ceil}((x_{in\_0} - F + 1)/S), 0) \\
\end{equation}
\begin{equation}
x_{out\_1} = \texttt{min}(\texttt{floor}((x_{in\_1} - 1)/S) + 1, W_2) \\
\end{equation}
\begin{equation}
y_{out\_0} = \texttt{max}(\texttt{ceil}((y_{in\_0} - F + 1)/S), 0) \\
\end{equation}
\begin{equation}
y_{out\_1} = \texttt{min}(\texttt{floor}((y_{in\_1} - 1)/S) + 1, W_2)\\
\end{equation}

\vspace{0.2in}

Also not that to compute these values in the output feature map we need to read a slightly bigger region from the input feature map due to the overlapping with the filter positions (marked in red dashed lines in Figure. \ref{fig:patch_propagation}). The coordinates of this read input patch, $x_{read\_0}\rightarrow x_{read\_1}$ in $x$ dimension and $y_{read\_0}\rightarrow y_{read\_1}$ in y dimension, is also a function filter size $F$ and stride $S$ of the Conv filter and can be expressed as follows:

\begin{equation}
x_{read\_0} = \texttt{max}(\texttt{ceil}((x_{in\_0} - F + 1)/S) \times S, 0) \\
\end{equation}
\begin{equation}
x_{read\_1} = \texttt{min}(\texttt{floor}((x_{in\_1} - 1)/S) \times S + F, W_1)\\
\end{equation}
\begin{equation}
y_{read\_0} = \texttt{max}(\texttt{ceil}((y_{in\_0} - F + 1)/S) \times S, 0) \\
\end{equation}
\begin{equation}
y_{read\_1} = \texttt{min}(\texttt{floor}((y_{in\_1} - 1)/S) \times S + F, W_2)\\
\end{equation}

\vspace{0.2in}

\begin{table}[t]
  \centering
  \caption{Notation used in Section. \ref{sec:ivm}}
  \scalebox{0.8}{\begin{tabular}{p{1cm}p{8.5cm}}
    \toprule
    \textbf{Symbol} & \textbf{Description}\\
    \midrule \midrule
    $W_1$ & Width of the input feature map to the Conv/Pool operator\\
    \midrule
    $D_1$ & Depth of the input feature map to the Conv operator\\
    \midrule
    $F$ & Width of filter kernel of the Conv/Pool operator\\
    \midrule
    $W_2$ & Width of the output feature map produced by the Conv/Pool operator\\
    \midrule
    $D_2$ & Depth of the output feature map produced by the Conv operator\\
    \midrule
    $x_{in\_0}$ & Starting x coordinate of the updated patch in the input feature map\\
    \midrule
    $x_{in\_1}$ & Ending x coordinate of the updated patch in the input feature map\\
    \midrule
    $y_{in\_0}$ & Starting y coordinate of the updated patch in the input feature map\\
    \midrule
    $y_{in\_1}$ & Ending y coordinate of the updated patch in the input feature map\\
    \midrule
    $x_{out\_0}$ & Starting x coordinate of the updated patch in the output feature map\\
    \midrule
    $x_{out\_1}$ & Ending x coordinate of the updated patch in the output feature map\\
    \midrule
    $y_{out\_0}$ & Starting y coordinate of the updated patch in the output feature map\\
    \midrule
    $y_{out\_1}$ & Ending y coordinate of the updated patch in the output feature map\\
    \midrule
    $x_{read\_0}$ & Starting x coordinate of the input feature map that need to be used for computing the updated output\\
    \midrule
    $x_{read\_1}$ & Ending x coordinate of the input feature map that need to be used for computing the updated output\\
    \midrule
    $y_{read\_0}$ & Starting y coordinate of the input feature map that need to be used for computing the updated output\\
    \midrule
    $y_{read\_1}$ & Ending y coordinate of the input feature map that need to be used for computing the updated output\\
    \bottomrule
  \end{tabular}}
\label{table:algo_symbols}
\vspace{-2mm}
\end{table}

\subsubsection{Estimating the maximum attainable theoretical speedup}

Important thing to notice with incremental inference of Conv and Pool layers for occlusion experiments is that the size of the updated patch in the output layer is larger than the updated patch in the input layer.
The growth is determined by the filter size and stride. Higher the filter size and stride higher the propagation rate of the modified patch.
With this observation, it is interesting to find out what percentage of computations can be saved by performing incremental inference of Conv and Pooling layers.
This can be easily estimated by iteratively calculating the updated patch sizes for Conv and Pooling layers based on the starting occlusion patch size $W_{patch}$ on the input image for all possible patch locations based on stride used $S_{patch}$.
Algorithm \ref{alg:max-speedup} shows how this can be calculated programmatically.
For sake of simplicity this algorithm assumes that the CNN architecture is a simple chained style architecture instead of more general style of directed-acyclic-graph (DAG).
However the algorithm can be easily extended support more general DAG style architectures.
It takes an object $CNN$ which is a nested information object containing information about different layers of the CNN and their properties, the size of the occlusion patch and the size of the stride for occlusion patch.
It then calculates the floating point operations required for incremental inference versus full inference for each possible location of the occlusion patch and computes the theoretical speedup which is the ratio between operation required for full inference and incremental inference.
It also computes the overall speedup which is the aggregation for all possible positions of the occlusion map and return this value along with 2D array containing individual position wise speedups as the output.

\begin{algorithm}
\caption{Estimate Maximum Theoretical Speedup}\label{alg:max-speedup}
\begin{algorithmic}[1]
\Procedure{EstimateMaxSpeedup}{$CNN$, $W_{patch}$, $S_{patch}$}
\State $flops_{inc} \gets 0$
\State $flops_{full} \gets 0$
\State $tmp \gets floor((CNN.image.W-W_{patch}+1)/S_{patch})$
\State $speedup \gets ARRAY[tmp][tmp]$

\For{\texttt{i in range(0, tmp, $S_{patch}$)}}
  \For{\texttt{i in range(0, tmp, $S_{patch}$)}}
      \State $tmp\_flops_{full} \gets 0$
      \State $tmp\_flops_{inc} \gets 0$
      
      \State $x_{in\_0} \gets i$
      \State $x_{in\_0} \gets i + W_{patch}$
      \State $y_{in\_0} \gets j$
      \State $y_{in\_0} \gets j + W_{patch}$

      \For{\texttt{k in range(0, size(CNN.layers), 1)}}
          \State $layer \gets CNN.layers[k]$
          \If {$layer.type$ $in$ $[conv, pool]$}
            \State $F \gets layer.filter.F$
            \State $S \gets layer.filter.S$
            
            \State $x_{out\_0} = \texttt{max}(\texttt{ceil}((x_{in\_0} - F + 1)/S), 0)$
            \State $x_{out\_1} = \texttt{min}(\texttt{floor}((x_{in\_1} - 1)/S) + 1, W_2)$
            \State $y_{out\_0} = \texttt{max}(\texttt{ceil}((y_{in\_0} - F + 1)/S), 0)$
            \State $y_{out\_1} = \texttt{min}(\texttt{floor}((y_{in\_1} - 1)/S) + 1, W_2)$

            \If {$layer.type$ $=$ $conv$}
              \State $W_1 \gets layer.input.W$
              \State $D_1 \gets layer.input.D$
              \State $W_2 \gets layer.output.W$
              \State $D_2 \gets layer.output.D$

              \State $tmp\_flops_{full} \mathrel{{+}{=}} F^2 \times D_1\times W_2^2\times D_2$
              \State $tmp\_flops_{inc} \mathrel{{+}{=}} F^2 \times D_1$
              \State $\hspace{10mm} \times ~(x_{out\_1} - x_{out\_0})$
              \State $\hspace{10mm} \times ~(y_{out\_1} - y_{out\_0})$
              \State $\hspace{10mm} \times ~D_2$
            \EndIf

            \State $(x_{in\_0}, x_{in\_1}, y_{in\_0}, y_{in\_1})$
            \State $\hspace{10mm} \gets (x_{out\_0}, x_{out\_1}, y_{out\_0}, y_{out\_1})$
          \ElsIf {$layer.type$ $=$ $fully-connected$}
            \State $W_1 \gets layer.input.W$
            \State $W_2 \gets layer.outputs.W$
            \State $tmp\_flops_{full} \mathrel{{+}{=}} W_1 \times W_2$
            \State $tmp\_flops_{inc} \mathrel{{+}{=}} W_1 \times W_2$
          \EndIf
      \EndFor

      \State $flops_{inc} \mathrel{{+}{=}} tmp\_flops_{inc}$
      \State $flops_{full} \mathrel{{+}{=}} tmp\_flops_{full}$
      \State $speedup[i][j] \gets tmp\_flops_{full}/tmp\_flops_{inc}$
      
  \EndFor
\EndFor

\State \textbf{return} ($flops_{full}$/$flops_{inc}$, $speedup$)
\EndProcedure
\end{algorithmic}
\end{algorithm}

We have applied this algorithm to analyze the theoretical maximum attainable speedup for many widely used CNN architectures by performing static analyzing on the CNN models defined using PyTorch framework (see Figure. \ref{fig:speedups}).
With an occlusion patch of size 16, moved with a stride of 1, for most CNN architectures we can achieve an average speedup of around 2.
However VGG and Squeezenet1\_0 CNN architectures can produce higher speedups than this.
The reason for this is VGG (16 and 19 layer versions) and Squeezenet1\_0 architectures use smaller Conv filter kernels $(3\times3)$. Therefore the rate of propagation of the occlusion patch is slower than other CNN architectures.
Thus more redundant computations can be saved from those architectures by applying incremental inference approach.

\begin{figure}
  \includegraphics[width=\columnwidth]{./images/speedup_plots}
  \caption{Maximum attainable theoretical speedup for incremental inference approach for an occlusion experiment with patch of size 16 pixels stride of 2}
  \label{fig:speedups}
\end{figure}

\subsubsection{Occlusion Experiment with Incremental Inference}
With all the necessary core ideas explained, we now explain how incremental inference approach can be used to implement occlusion experiment (see Algorithm. \ref{alg:inc-inference}).
On a high-level the structure of this algorithm is similar to the theoretical speedup calculation algorithm. But the following differences can be noted. The algorithm takes an input image $I$ as input for the occlusion experiment in addition to the CNN model, patch width $W_{patch}$, and stride $S_{patch}$.
It then performs a full inference on the image and obtain a list containing activations for all the layers, including input image, by calling the method \texttt{PerformFullCNNInference(CNN, I)}.
It then finds the index of the predicted class label by finding the index of the maximum activation in the softmax layer, which is the last layer in the CNN.
Next similar to Algorithm. \ref{alg:max-speedup}, it iterates through all the possible positions of the occlusion patch on the input image and perform incremental inference for Conv and Pool layers based on the updated patch locations. For fully-connected and softmax layers usual full-inference is performed as there is no redundancy in computations.
When performing incremental inference for Conv and Pooling layers, the updated output of the previous layers is stitched together with pre-materialized values obtained from full inference ($M$) corresponding to that layer to create the input patch for the incremental inference operator.

Similar to speedup calculations, for simplicity, the algorithm here assumes the CNN architecture is a simple chain styled architecture. However it can be easily extended to support more general DAG style CNN architectures.
Another point to notice is that this algorithm performs inference for single occlusion position at a time. Alternatively one could batch together multiple occlusion patch positions together and perform batched inference. However since the patch sizes are not guaranteed to be the same at each layer, they may be need to be padded to transform them into the same size.
Batching multiple inferences together can reduce the runtime of occlusion experiments as it can amortize the overheads specially when using GPUs for inference.

\begin{algorithm}
\caption{Occlusion Experiment with Incremental Inference}\label{alg:inc-inference}
\begin{algorithmic}[1]
\Procedure{OcclusionWithIncrementalInference}{$I$,$CNN$, $W_{patch}$, $S_{patch}$}

\State $M \gets \texttt{PerformFullCNNInference}(CNN, I)$
\State $label_{index} \gets argmax(M[-1])$
\State $tmp \gets floor((CNN.image.W-W_{patch}+1)/S_{patch})$
\State $P \gets ARRAY[tmp][tmp]$

\For{\texttt{i in range(0, tmp, $S_{patch}$)}}
  \For{\texttt{i in range(0, tmp, $S_{patch}$)}}

      \State $x_{in\_0} \gets i$
      \State $x_{in\_0} \gets i + W_{patch}$
      \State $y_{in\_0} \gets j$
      \State $y_{in\_0} \gets j + W_{patch}$

      \State $x \gets Zero[x_{in\_1}-x_{in\_0}][y_{in\_1}][y_{in\_0}]$
      
      \For{\texttt{k in range(0, size(CNN.layers), 1)}}
          \State $layer \gets CNN.layers[k]$
          \If {$layer.type$ $in$ $[conv, pool]$}

            \State $x_{read\_0} \gets \texttt{max}(\texttt{ceil}((x_{in\_0} - F + 1)/S)$
            \State $\hspace{10mm} \times ~S, 0)$
            \State $x_{read\_1} \gets \texttt{min}(\texttt{floor}((x_{in\_1} - 1)/S)$
            \State $\hspace{10mm} \times ~S + F, W_1)$
            \State $y_{read\_0} \gets \texttt{max}(\texttt{ceil}((y_{in\_0} - F + 1)/S)$
            \State $\hspace{10mm} \times ~S, 0)$
            \State $y_{read\_1} \gets \texttt{min}(\texttt{floor}((y_{in\_1} - 1)/S)$
            \State $\hspace{10mm} \times ~S + F, W_2)$

            \State $tmp \gets M[k]$
            \State $tmp[x_{in\_0}:x_{in\_1}][y_{in\_0}:y_{in\_1}] \gets x$

            \State $x \gets tmp[x_{read\_1}-x_{read\_0}]$
            \State $\hspace{20mm}       [y_{read\_1}-y_{read\_0}]$

            \State $x \gets layer.transform(x)$

            \State $F \gets layer.filter.F$
            \State $S \gets layer.filter.S$
            
            \State $x_{out\_0} = \texttt{max}(\texttt{ceil}((x_{in\_0} - F + 1)/S), 0)$
            \State $x_{out\_1} = \texttt{min}(\texttt{floor}((x_{in\_1} - 1)/S) + 1, W_2)$
            \State $y_{out\_0} = \texttt{max}(\texttt{ceil}((y_{in\_0} - F + 1)/S), 0)$
            \State $y_{out\_1} = \texttt{min}(\texttt{floor}((y_{in\_1} - 1)/S) + 1, W_2)$
            
            \State $(x_{in\_0}, x_{in\_1}, y_{in\_0}, y_{in\_1})$
            \State $\hspace{10mm} \gets (x_{out\_0}, x_{out\_1}, y_{out\_0}, y_{out\_1})$
          \ElsIf {$layer.type$ $=$ $fully-connected$}
            \State $tmp \gets CNN.layers[k-1].type$
            \If {$tmp \mathrel{{!}{=}} fully-connected$}
              \State $tmp \gets M[k-1]$
              \State $tmp[x_{out\_0}:x_{out\_1}]$$[y_{out\_0}:y_{out\_1}]$$\gets x$
              \State $x \gets tmp$
            \EndIf
            \State $x \gets fully-connected(x, layer.weights)$
          \ElsIf {$layer.type$ $=$ $softmax$}
            \State $x \gets softmax(x)$
          \EndIf
      \EndFor

      \State $P[i][j] \gets x[label_{index}]$
      
  \EndFor
\EndFor

\State \textbf{return} $(label_{index}, P)$
\EndProcedure
\end{algorithmic}
\end{algorithm}

\section{Early Experimental Results}
In this section we summarize the early experimental results obtained by implementing the incremental inference for occlusion experiments.

\textbf{Experimental Setup.} The experiments was performed on an Intel(R) Core(TM) i7-6700 CPU 3.40GHz machine with 32 GB RAM. The machine is also equipped with a Nvidia Titan Xp GPU. We use PyTorch 0.3.1 library as the deep learning toolkit library.

\textbf{Workload}. We use a popular ImageNet pre-trained VGG 16 layer CNN model and subject it to occlusion experiments. The performance of the occlusion experiment with naive approach and incremental inference approach is benchmarked on CPU and GPU separately.
An occlusion patch of size $16\times16$ was placed on the center of the image of size $224\times224$ and runtime for full inference approach and incremental inference approach average over 5 iterations is recorded. From these values the speedup is calculated. The experiment was repeated with a batch size of 1 and 16 (see Table. \ref{table:speedup}).

\begin{table}[h]
\centering
\begin{tabular}{|l|l|l|l|}
\hline
\multirow{2}{*}{Batch Size} & \multirow{2}{*}{Theoretical Speedup} & \multicolumn{2}{l|}{Experimental Speedup} \\ \cline{3-4} 
 &  & CPU & GPU \\ \hline
1 & 7.6 & 5.4 & 1.3 \\ \hline
16 & 7.6 & 5.4 & 1.6 \\ \hline
\end{tabular}
\caption{Theoretical versus empirical speedup achievable with incremental inference}
\label{table:speedup}
\end{table}
\vspace{-5.mm}

Our implementation of incremental inference of Conv and Pool layers could achieve a speedup of 5.4 on CPU for a batch size of 1 and 16 compared to the theoretical maximum speedup of $7.4$. However, the GPU implementation could achieve a speedup of only 1.3 and this can be increased upto 1.6 with a batch size of 16. We suspect that the random memory operations introduced by the incremental inference approach due to stitching of output of incremental Conv and Pool layers with pre-materialized activations of the full inference is  throttling the GPU performance in incremental approach.

\section{Conclusions \& Future Work}
In this work explore applying incremental inference of Conv and Pooling layers of CNN models to reduce the runtime of occlusion experiments.
We formalize the incremental inference approach for CNNs and evaluate the theoretical upper-bound of speedup achievable for different CNN architectures by statically analyzing the CNN architecture definitions specified in PyTorch framework.
For most CNN architectures we can achieve a speedup higher than 2 and for some CNN architectures, such as VGG and Squezenet1\_0, this can be higher than $7$.
Our implementation of incremental inference of Conv and Pool layers could achieve a empirical speedup of 5.4 on CPU and 1.6 on GPU with a batch size of 16. We suspect the relatively low speedup of GPU implementation is attributable to the random memory operations caused by incremental inference throttling the GPU performance.

As future work we are looking in to developing a more efficient GPU implementation which can attain higher speedup. This will require implementing GPU kernels which can perform incremental Conv and Pooling operations in place without additional memory copying.
We also plan to explore several other optimizations including explore and exploit style approaches on localizing the most sensitive image regions and approximate CNN inference for reducing the runtime of occlusion based CNN explainability workloads.
Explore and exploit style approach is an algorithmic optimization motivated by the specific use cases of occlusion experiments.
In most applications such as medical imaging the objects of interest in an image occupies a relatively small portion and are located together.
In such settings rather using a patch of small size we can start with a large patch with a relatively large stride and then iteratively focus into smaller regions which appears to be sensitive for the predicted class label.
% We name this approach as \textit{hierarchical inference of occlusion patches}.
Approximate inference of CNN approach is based on a general observation of deep CNN inference.
Even though in theory the projective field of an input pixel grows linearly in practice the effective projective field does not grow in that rate.
This mean most of the local changes in the input space are affecting localized changes in the output space of a convolution operation.
Therefore we plan to experiment the possibility of constraining the growth of the projective field of an input pixel and there by reduce runtime using our \textit{incremental inference} approach.
% We name this optimization as \textit{approximate inference} of deep CNNs.

\section{Other Related Work}

\section{Conclusions And Future Work}

\bibliographystyle{unsrt}
\bibliography{main}

%!TEX root = <main.tex>
\appendix

\end{document}